\chapter{Практические задания}
\section{Написать хвостовую рекурсивную функцию my-reverse, которая развернет верхний уровень своего списка-аргумента lst.}

\begin{lstlisting}[basicstyle=\footnotesize, caption={Задание 1}]
(defun my_reverse (lst)
	(move_to_front lst ())
)

(defun move_to_front (lst result)
	(cond ((null lst) result)
		  (T (move_to_front (cdr lst) (cons (car lst) result)) )
	)
)
\end{lstlisting}

\section{Написать функцию, которая возвращает первый элемент списка-аргумента, который сам является непустым списком.}
\begin{lstlisting}[basicstyle=\footnotesize, caption={Задание 2}]
(defun return_first_list (lst)
	(cond ((null lst) nil) 
		  ((not (and (listp (car lst)) (> (length (car lst)) 0) )) (return_first_list (cdr lst)) )
		  (T (car lst))
	)
)
\end{lstlisting}

\newpage
\section{Написать функцию, которая выбирает из заданного списка только те числа, которые больше 1 и меньше 10.}
\begin{lstlisting}[basicstyle=\footnotesize, caption=Задание 3]
(defun return_in_range (lst)
	(cond ((null lst) nil)
		  ((and (> (car lst) 1) (< (car lst) 10)) (cons (car lst) (return_in_range (cdr lst))))
		  (T (return_in_range (cdr lst)))
	)
)
\end{lstlisting}

\section{Напишите рекурсивную функцию, которая умножает на заданное число-аргумент все числа из заданного списка-аргумента, когда a) все элементы списка -- числа, 6) элементы списка -- любые объекты.}
\begin{lstlisting}[basicstyle=\footnotesize, caption=Задание 4]
(defun multiply_by_number_1 (lst number)
	(cond ((null lst) nil)
		  ((cons (* number (car lst)) (multiply_by_number_1 (cdr lst) number) ))
	)
)

(defun multiply_by_number_2 (lst number)
	(cond ((null lst) nil)
		  ((numberp (car lst)) (cons (* number (car lst)) (multiply_by_number_2 (cdr lst) number)))
		  (T (cons (car lst) (multiply_by_number_2 (cdr lst) number)))
	)
)
\end{lstlisting}

\newpage
\section{Напишите функцию, select-between, которая из списка-аргумента, содержащего только числа, выбирает только те, которые расположены между двумя указанными границами-аргументами и возвращает их в виде списка (упорядоченного по возрастанию списка чисел.}
\begin{lstlisting}[basicstyle=\footnotesize, caption={Задание 5}]
(defun select_between (lst board_left board_right)
	(bubble_sort_asc (select lst board_left board_right))
)

(defun select (lst board_left board_right)
	(cond ((null lst) nil)
		  ((and (>= (car lst) board_left) (<= (car lst) board_right)) 
		  (cons (car lst) (select (cdr lst) board_left board_right))
		  )
		  (T (select (cdr lst) board_left board_right))
	)
)

(defun bubble_sort_asc (lst)
	(cond ((atom (cdr lst)) lst)
	 	   (T (bubble (cons (car lst) (bubble_sort_asc (cdr lst)))))
	)
)

(defun bubble (lst)
	(cond ((atom (cdr lst)) lst)
		  ((> (car lst) (cadr lst)) 
		  (cons (cadr lst) (bubble (cons (car lst) (cddr lst)) ))
		  )
		  (T lst)
	)
)
\end{lstlisting}

\section{Написать рекурсивную версию (с именем rec-add) вычисления суммы чисел заданного списка: а) одноуровнего смешанного, б) структурированного.}
\begin{lstlisting}[basicstyle=\footnotesize, caption=Задание 6]
(defun rec_add_1 (lst)
	(let ((sum 0))
		 (calc_sum lst sum)
	)
)

(defun calc_sum (lst sum)
	(cond ((null lst) sum)
		  ((numberp (car lst)) (calc_sum (cdr lst) (+ sum (car lst))))
		  (T (calc_sum (cdr lst) sum))
	)
)

(defun rec_add_2 (lst)
	(calc_sum_structure_lst lst 0)
)

(defun calc_sum_structure_lst (lst sum)
	(cond ((null lst) sum)
		  ((numberp (car lst)) (calc_sum_structure_lst (cdr lst) (+ sum (car lst))))
		  ((listp (car lst)) (+ (calc_sum_structure_lst (car lst) sum) (calc_sum_structure_lst (cdr lst) 0)))
		  (T (calc_sum_structure_lst (cdr lst) sum))
	)
)
\end{lstlisting}

\newpage
\section{Написать рекурсивную версию с именем recnth функции nth.}
\begin{lstlisting}[basicstyle=\footnotesize, caption={Задание 7}]
(defun rec_nth (lst position)
	(cond ((not (check_position lst position)) nil)
 		   (T (find_elem lst position 0))
	)
)

(defun check_position (lst position)
	(cond ((> position (- (length lst) 1)) nil)
		  (T T)
	)
)

(defun find_elem (lst position index)
	(cond ((= index position) (car lst))
		  (T (find_elem (cdr lst) position (+ index 1)))
	)
)
\end{lstlisting}

\section{Написать рекурсивную функцию allodd, которая возвращает t когда все элементы списка нечетные.}
\begin{lstlisting}[basicstyle=\footnotesize, caption=Задание 8]
(defun all_odd (lst)
	(cond ((null lst) T)
		  ((oddp (car lst)) (all_odd (cdr lst)))
	 	  (T nil)
	)
)
\end{lstlisting}

\newpage
\section{Написать рекурсивную функцию, которая возвращает первое нечетное число из списка (структурированного).}
\begin{lstlisting}[basicstyle=\footnotesize, caption=Задание 9]
(defun find_first_odd (lst)
	(cond ((null lst) nil)
		  ((numberp (car lst)) (cond ((oddp (car lst)) (car lst))
		  							 (T (find_first_odd (cdr lst))))
		  )
		  ((listp (car lst)) (or (find_first_odd (car lst)) (find_first_odd (cdr lst))))
		  (T (find_first_odd (cdr lst)))
	)
)
\end{lstlisting}

\section{Используя cons-дополняемую рекурсию с одним тестом завершения, написать функцию которая получает как аргумент список чисел, а возвращает список 	квадратов этих чисел в том же порядке.}
\begin{lstlisting}[basicstyle=\footnotesize, caption=Задание 10]
(defun square_lst_elems (lst)
	(cond ((null lst) nil)
		  (T (cons (* (car lst) (car lst)) (square_lst_elems (cdr lst))))
	)
)
\end{lstlisting}