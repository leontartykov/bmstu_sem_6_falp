\chapter{Теоретические вопросы}
\section{Структуроразрушающие и не разрушающие структуру списка функции.}
\textit{Не разрушающие структуру списка функции} -- функции, которые не изменяют исходный список. Примеры таких функций:
\begin{enumerate}
	\item \textbf{append} -- имеет переменной число параметров; выполняет объединение списков в один; возвращает список в качестве значения; выполняет копирование всех элементов списка, кроме последнего.
	\item \textbf{remove} -- имеет два аргумента; удаляет все вхождения элемента el из списка lst (формула \ref{eq:remove}).
	
	\begin{equation}
		\label{eq:remove}
		(remove\ el\ lst)
	\end{equation}

	\item{\textbf{reverse}} -- имеет один аргумент и меняет порядок элементов аргумента на противоположный.
	\item{\textbf{substitute}} -- имеет три аргумента; заменяет все элементы списка, которые равны второму аргументу на значение первого.
	\item{\textbf{member}} -- имеет два аргумента; возвращает хвост списка, начиная со списковой ячейки, удовлетворяющей первому аргументу; иначе возвращает nil
	\item{\textbf{nthcdr}} -- имеет два аргумента (номер элемента (n) и список) ; возвращает хвост списка, начиная с n-ого элемента.
	\item{\textbf{nth}} -- имеет два аргумента (номер элемента (n), список); возвращает car-указатель на n-ый элемент списка.
	\item{\textbf{length}} -- имеет один аргумент (последовательность -- строки или список); возвращает число элементов второго аргумента.
\end{enumerate}

\textit{Структуроразрушающие функции} -- функции, которые изменяют указатели списка. Таким образом, не происходит копирования элементов списка. Такие функции обычно начинаются с символа n (nreverse, nconc, ...). Примеры таких функций:
\begin{enumerate}
	\item \textbf{nconc} -- аналогично функции append, но копирования элементов не происходит.
	\item \textbf{delete} -- аналогично функции remove; не создает копии исходного списка.
	\item \textbf{nreverse} -- аналогично функции reverse; не создает копии исходного списка.
	\item \textbf{nsubstitute} -- аналогично функции substitute; не создает копии исходного списка.
\end{enumerate}



\section{Отличие в работе функций cons, list, append, nconc и в их результате.}
\textit{cons} -- имеет два аргумента и возвращает бинарный узел. Если вторым аргументом является атом, то возвращается точечная пара; если список -- список.

\textit{list} -- имеет произвольное число аргументов и возвращает список.

\textit{append} -- имеет произвольное число аргументов; важным свойством является то, что создается копия всех аргументов, кроме последнего, и дальнейшая работа ведется с ней; при этом сохраняется возможность работать с исходным списком. Таким образом, при изменени. В результате функции возвращается список.

\textit{nconc} -- функция, аналогичная append за исключением того, что исходный список не копируется (структуроразрушающая функция).