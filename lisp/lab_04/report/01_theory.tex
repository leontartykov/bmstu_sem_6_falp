\chapter{Теоретические вопросы}
\section{Синтаксическая форма и хранение программы в памяти}
В Lisp программы и данные синтаксически выглядят в виде \textit{S-выражения}. S-выражение может быть как атомом, который в памяти представляется в виде 5 указателей (name, value, function, properties, package), так и точечной парой -- 2 указателя (бинарный узел).

\section{Трактовка элементов списка}
Элементы списка представляются следующим образом: первый - имя функции, остальные -- её аргументы. Формат списка представлен в виде формулы (\ref{eq:list}).
\begin{equation}
	\label{eq:list}
	(\textup{функция аргумент\_1}\ldots\text{аргумент\_n}) , n \geq 0
\end{equation}

\section{Порядок реализации программы}
Набранные S-выражения выполняются при помощи интерпретатора -- функцией eval, после выполнения которой возвращается полученный результат.



\section{Способы определения функции}
\begin{enumerate}
	\item lambda-выражение. Данный способ представлен в виде формулы (\ref{eq:lambda}).
	
	\begin{equation}
		\label{eq:lambda}
		(lambda \ \lambda \text{-} \textup{список} \ \textup{форма}),
	\end{equation}
	
	где $\lambda$-$\textup{список}$ -- список формальных параметров, $\textup{форма}$ -- тело функции.
	
	lambda$\text{-} \textup{выражение}$ не хранится в памяти и не имеет имени. Вычисляется сразу же. Используется для повторных вычислений.
	
	Вызов lambda$\text{-} \textup{функции}$ выполняется по формуле (\ref{eq:lambda_function}).
	
	\begin{equation}
		\label{eq:lambda_function}
		(\lambda \text{-} \textup{выражение} \ \textup{последовательность форм})
	\end{equation}
	
	\item С помощью \textit{defun} по формуле (\ref{eq:defun}). 
	
	\begin{equation}
		\label{eq:defun}
		(defun \ f \ \lambda \text{-} \textup{выражение})
	\end{equation}
	
	Система по имени символьного атома находит его определение.
\end{enumerate}

\section{Принципиальное отличие функций cons, list, \\ append}
\textit{cons} -- имеет два аргумента и возвращает бинарный узел. Если вторым аргументом является атом, то возвращается точечная пара; если список -- список.

\textit{list} -- имеет произвольное число аргументов и возвращает список.

\textit{append} -- имеет произвольное число аргументов; важным свойством является то, что создается копия всех аргументов, кроме последнего. В результате функции возвращается список.

\begin{lstlisting}[label=lst:cons, caption=Использование cons.]
	(cons 'A 'B)          ;; (A.B)
	(cons 'A '(B C D))    ;; (A B C D)
	(cons '(A B) '(C D))  ;; ((A B) C D) 
\end{lstlisting}

Пример использования list представлен на листинге \ref{lst:list}.
\begin{lstlisting}[label=lst:list, caption=Использование list.]
	(list 'A 'B)         ;; (A B)
	(list 'A '(B C) 'D)  ;; (A (B C) D)
\end{lstlisting}

\newpage
Функция list может быть представлена с помощью cons (листинг \ref{lst:list_cons}).

\begin{lstlisting}[label=lst:list_cons, caption=Представление list с помощью cons.]
	(defun list2 (ar1 ar2) (cons ar1 (cons ar2 ())))
	(defun list3 (ar1 ar2 ar3) (cons ar1 (cons ar2 (cons ar3 ()))))
	...
\end{lstlisting}

Отличие в использовании \text{cons}, \text{list}, \text{append} представлен на листинге \ref{lst:conslistappend}.
\begin{lstlisting}[label=lst:conslistappend, caption=Использование \text{cons,} \text{list,} \text{append}.]
	(list '(a b) '(c d)) 	;((a b) (c d))
	(cons '(a b) '(c d))    ;((a b) c d)
	(append '(a b) '(c d))  ;(a b c d)
\end{lstlisting}

