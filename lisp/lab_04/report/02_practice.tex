\chapter{Практические задания}
\section{Каковы результаты вычисления следующих выражений?}
\begin{lstlisting}
	(setf lst1 '(a b))
	(setf lst2 '(c d))
	
	(cons lst1 lst2)   ; ((a b) c d) 
	(list lst1 lst2)   ; ((a b) (c d))
	(append lst1 lst2) ; (a b c d)
\end{lstlisting}
\section{Каковы результаты вычисления следующих выражений?}
\begin{lstlisting}
	(reverse ()) 			; nil
	(last ())				; nil
	(reverse '(a))			; (a)
	(last '(a))				; (a)
	(reverse '((a b c)))	; ((a b c))
	(last '((a b c)))		; ((a b c))
\end{lstlisting}

\section{Написать, по крайней мере, два варианта функции, которая возвращает последний элемент своего списка-аргумента.}
\begin{lstlisting}
	(defun last_elem_1 (arg) (car (last arg)))
	
	(defun last_elem_2 (arg) (first (nreverse arg)))
\end{lstlisting}

\section{Написать, по крайней мере, два варианта функции, которая возвращает свой список-аргумент без последнего элемента.}
\begin{lstlisting}
	(defun without_last_1 (arg) (nreverse (cdr (nreverse arg))))
	
	(defun without_last_2 (arg) 
		(reverse (last (nreverse arg) (- (length arg) 1))))
\end{lstlisting}

\section{Игра в кости}
Простой вариант игры в кости, в котором бросаются две правильные кости. Если
сумма выпавших очков равна 7 или 11 -- выигрыш, если выпало (1,1) или (6,6) --- игрок право снова бросить кости, во всех остальных случаях ход переходит ко второму игроку, но запоминается сумма выпавших очков. Если второй игрок не выигрывает абсолютно, то выигрывает тот игрок, у которого больше очков. Результат игры и значения выпавших костей выводить на экран с помощью функции print.

\begin{lstlisting}[basicstyle=\footnotesize,caption= Программный код <<Игра в кости>>]
(defvar first_gamer nil)
(defvar second_gamer nil)

(defun roll_dice()
	(+ (random 6) 1))

(defun is_winner (first_gamer_numbers)
	(cond ((= (+ (car first_gamer_numbers) (cadr first_gamer_numbers)) 7) T)
		  ((= (+ (car first_gamer_numbers) (cadr first_gamer_numbers)) 11) T)
    )
)

(defun print_gamer(gamer value_gamer)
	(print gamer)
	(print (car value_gamer))
)
\end{lstlisting}
\newpage

\begin{lstlisting}[basicstyle=\footnotesize,caption= Программный код <<Игра в кости>> (продолжение)]
(defun is_new_move (first_gamer_numbers)
   (cond (
   		  (and (= (car first_gamer_numbers) 1) (= (cadr first_gamer_numbers) 1))
		   'one
		 )
		 (
		  (and (= (car first_gamer_numbers) 6) (= (cadr first_gamer_numbers) 6)) 'two
		 )
    )
)

(defun choose_winner()
	(or 
		(if (> (+ (caar first_gamer) (cadar first_gamer)) 
			   (+ (caar second_gamer) (cadar second_gamer)))
			(print_winner "Первый игрок победил" 1) (print_winner "Второй игрок победил" 1)
		)
		(print_winner "Ничья")
	)
)

(defun print_winner(str_win type_win)    
	(print "Игра_завершена")
	(print_gamer "Первый игрок" first_gamer)
	(if (null second_gamer) (print "Второй игрок не сделал ход.") 
		(print_gamer "Второй игрок" second_gamer))
	(if (= 1 type_win) (print str_win) (choose_winner))
)

(defun user_round (gamer)
	(setq gamer (list (roll_dice) (roll_dice)))
	(cond ((is_winner gamer) (list gamer 1))
		  ((is_new_move gamer) (user_round gamer))
		  (T (list gamer 0)) 
	)
)
\end{lstlisting}
\newpage

\begin{lstlisting}[basicstyle=\footnotesize,caption= Программный код <<Игра в кости>>(продолжение)]
(defun play_game()
	(setq first_gamer (user_round first_gamer))
	(or (cond ((= 1 (cadr first_gamer))
			   (print_winner "Абсолютная победа первого игрока" 1)))
		(and (setq second_gamer (user_round second_gamer))
			 (cond ((= 1 (cadr second_gamer)) 
			 		(print_winner "Абсолютная победа второго игрока" 1)))
		)
		(choose_winner)
	)
)        

(play_game)
\end{lstlisting}

