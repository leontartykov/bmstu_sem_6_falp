\chapter{Практические задания}
\section{Напишите функцию, которая уменьшает на 10 все числа из списка-аргумента этой функции.}

\begin{lstlisting}[basicstyle=\footnotesize, caption={Задание 1}]
(defun subtract_by_ten (lst)
	(mapcar #'(lambda (x) (- x 10)) lst)
)
\end{lstlisting}

\section{Напишите функцию, которая умножает на заданное число-аргумент все числа из заданного списка-аргумента, когда 	a) все элементы списка --- числа, 	б) элементы списка -- любые объекты.}
\begin{lstlisting}[basicstyle=\footnotesize, caption={Задание 2}]
(defun multiply_by_number_1 (lst number)
	(mapcar #'(lambda (x) (* x number)) lst)
)

(defun multiply_by_number_2 (lst number)
	(mapcar #'(lambda (x) (if (numberp x) (* x number) 
							  (if (listp x) 
							      (multiply_by_number_1 x number) 
							      x
							  ) 
						   )
			   ) lst)
)
\end{lstlisting}

\newpage
\section{Написать функцию, которая по своему списку-аргументу lst определяет является ли он палиндромом.}
\begin{lstlisting}[basicstyle=\footnotesize, caption=Задание 3]
(defun is_palindrom (lst)
	(let ((reverse_lst (reverse lst))
		  (result nil)
		 )
		 (setq result (mapcar #'(lambda (x y) (equal x y)) lst reverse_lst))
		 (every #'(lambda (x) (not (null x))) result)
	)
)
\end{lstlisting}

\section{Написать предикат set-equal, который возвращает t, если два его множества-аргумента содержат одни и те же элементы, порядок которых не имеет значения.}
\begin{lstlisting}[basicstyle=\footnotesize, caption=Задание 4]
(defun set_equal (lst1 lst2)
	(let ((result nil))
		  (cond ((/= (length lst1) (length lst2)) nil)
				 (T (setq result 
					(maplist #'(lambda (x) (if (member (car x) lst2) T nil)) lst1)
					)
				 )
		   )
		   (cond ((not (null result)) (every #'(lambda (x) (eql x T)) result)))
	)
)
\end{lstlisting}

\newpage
\section{Написать функцию которая получает как аргумент список чисел, а возвращает список квадратов этих чисел в том же порядке.}
\begin{lstlisting}[basicstyle=\footnotesize, caption={Задание 5}]
(defun square_numbers (lst)
	(mapcar #'(lambda (x) (* x x)) lst)
)
\end{lstlisting}

\section{Напишите функцию, select-between, которая из списка-аргумента, содержащего только числа, выбирает только те, которые расположены между двумя указанными границами-аргументами и возвращает их в виде списка (упорядоченного по возрастанию списка чисел).}
\begin{lstlisting}[basicstyle=\footnotesize, caption=Задание 6]
(defun select_between (lst board_left board_right)
	(mapcan #'(lambda (x) 
			  (and (>= x board_left) (<= x board_right) (list x)))
			  lst)
)

(defun bubble (lst)
	(cond ((atom (cdr lst)) lst)
		  ((> (car lst) (cadr lst)) 
		  (cons (cadr lst) (bubble (cons (car lst) (cddr lst)) ))
		  )
		  (T lst)
	)
)

(defun bubble_sort_asc (lst)
	(cond ((atom (cdr lst)) lst)
		   (T (bubble (cons (car lst) (bubble_sort_asc (cdr lst)))))
	)
)
\end{lstlisting}

\section{Написать функцию, вычисляющую декартово произведение двух своих списков-аргументов.}
\begin{lstlisting}[basicstyle=\footnotesize, caption={Задание 7}]
(defun cartesian_product (lst1 lst2)
	(mapcan #'(lambda (x) (mapcar #'(lambda (y) (list x y)) lst2)) lst1)
)
\end{lstlisting}

\section{Почему так реализовано reduce, в чем причина?}
\begin{lstlisting}[basicstyle=\footnotesize, caption=Задание 8]
	(reduce #'+ 0)   ; 0
	(reduce #'+ ())  ; 0
\end{lstlisting}
Функционал reduce имеет вызов по формуле (\ref{eq:reduce}):
\begin{equation}
	\label{eq:reduce}
	(reduce\ \#'fun\ lst)
\end{equation}
Reduce сначала применяет функцию fun к первым двум элементам списка; затем fun применяется к полученному результату и следующему элементу. Пример приведен на листинге \ref{lst:reduce}.
\begin{lstlisting}[basicstyle=\footnotesize, label=lst:reduce]
	(reduce #'- '(1 2 3 4)) ; (- (- (- 1 2) 3) 4) =>  -8
\end{lstlisting}
Данный функционал может содержать начальное значение (\textit{initial-value}) для некоторых операторов. Возможны два случая.
\begin{enumerate}
	\item Список lst содержит ровно один элемент и начальное значение не задано; тогда этот элемент возвращается, а функция fun не вызывается.
	\item Если список lst пуст, и задано начальное значение, то возвращается начальное значение, а функция не вызывается.
\end{enumerate}

Пример данных ситуаций приведен на листинге \ref{lst:examples_reduce}.
\begin{lstlisting}[basicstyle=\footnotesize, label=lst:examples_reduce]
	(reduce #'+ '())   ; 0 (1)
	(reduce #'+ '(3))  ; 3 (2)
	(reduce #'* '(2))  ; 2 (3)
\end{lstlisting}
Пример (1) удовлетворяет случаю (2); в итоге возвращается начальное значение оператора <<+>> 0. Случаю (1) удовлетворяет пример (2) и (3). 

\section{Пусть list-of-list список, состоящий из списков. Написать функцию, которая вычисляет сумму длин всех элементов list-of-list.}
\begin{lstlisting}[basicstyle=\footnotesize, caption=Задание 9]
(defun list_of_list (lst)
	(reduce #'(lambda (count x) (+ count (length x))) (cons 0 lst))
)
\end{lstlisting}