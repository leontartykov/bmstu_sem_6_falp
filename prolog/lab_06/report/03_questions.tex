\chapter{Контрольные вопросы}
\begin{enumerate}
	\item \textbf{Что такое рекурсия? Как организуется хвостовая рекурсия в Prolog? Как организовать выход из рекурсии?} \\
	Рекурия -- ссылка на описываемый объект при описании объекта. Хвостовая рекурсия организовывается следующим образом: сначала выполняются необходимые вычисления, и последним шагом такой <<функции>> является вызов того же самого объекта. При этом вычисления собираются по мере выхода из рекурсии. Для того, чтобы система не выполняла лишних действий и при этом правильно отрабатывала, необходимо ставить условия выхода из рекурсии вначале.
	
	\item \textbf{Какое первое состояние резольвенты?} \\	
	В резольвенте изначально хранится конъюнкция вопросов.
	
	\item \textbf{В каком случае система запускает алгоритм унификации? Каково назначение использования алгоритма унификации? Каков результат алгоритма унификации?} \\
	Назначение алгоритма унификации -- подбор знаний. Результатом её работы является ответ <<да>> или <<нет>>, т.е удалось ли ей подобрать знание или нет.
	
	\item \textbf{В каких пределах программы переменные уникальны?} \\
	Именованные переменные уникальны в пределах одного предложения, анонимные переменные -- уникальные всегда.
	
	\item \textbf{Как применяется подстановка, полученная с помощью алгоритма унификации?} \\
	Подстановка применяется путем конкретизации переменных.
	
	\item \textbf{Как изменяется резольвента?} \\
	Резольвента меняется в два этапа.
	\begin{enumerate}
		\item Новое состояние приобретается в результате алгоритма редукции. 
		\item К полученному состоянию применятся подстановка.
	\end{enumerate}

	\item \textbf{В каких случаях применяется механизм отката?} \\
	Механизм отката применяется в случае тупиковой ситуации -- в ситуации, когда нельзя перейти и данного состояния в новое, и при этом резольвента непуста.
\end{enumerate}