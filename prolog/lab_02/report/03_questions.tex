\chapter{Контрольные вопросы}
Лабораторная работа  №2 (часть 1):
\section{Что такое терм?} 
См. пункт 1.1.
\section{Что такое предикат в матлогике?}
Предикат -- суждение, которое становится высказыванием при конкретизации параметров. Некое отображение, которое каждому набору входящих в него параметров ставит в соответствие высказывание.
\section{Что описывает предикат в Prolog?}
В Prolog предикат описывает структуру знания и не связано с типизацией и распределением памяти.
\section{Назовите виды предложений в программе и приведите примеры таких предложений из вашей программы? Какие предложения являются основными, какие -- неосновными? Каковы их синтаксис и семантика (основных и неосновных)?}
Предложения могут быть: факты или правила.

\section{Каковы назначение, виды и особенности использования переменных в программе на Prolog? Какое предложение БЗ сформулировано в более общей абстрактной форме: содержащее или не содержащее переменных?}
Ответ на первый вопрос: см пункт 1.1.

Переменные служат для повышения уровня абстракции, поэтому неосновное предложение сформулировано в более общей абстрактной форме.
\section{Что такое подстановка?}
См. пункт 1.4.
\section{Что такое пример терма? Как и когда строится? Как вы думаете, система строит и хранит примеры?}
Определение терма см. пункт 1.4. 

Лабораторная работа №2 (часть 2):
\section{В какой части правила сформулировано знание? Это знание о чем, с формальной точки зрения?}
Знание сформулировано в заголовке правила.
\section{Что такое процедура?}
Процедура - совокупность правил, заголовки которых имеют одинаковые функторы, одинаковое количество аргументов, обозначающие объекты одной и той же природы.
\section{Сколько в БЗ текущего задания процедур?}
\section{Что такое пример терма, это частный случай терма, пример? Как строится пример?}
\section{Что такое наиболее общий пример?}
Определение наиболее общего терма см. пункт 1.4.
\section{Назначение и результат работы алгоритма унификации. Что значит двунаправленная передача параметров при работе алгоритма унификации. Пояснить на примере одного из случаев пункта 3.}
Алгоритм унификаций необходим для подбора знаний. Результатом будет определение похожести термов по смыслу.
\section{В каком случае запускаем механизм отката?}
Механизм отката запускается, когда или найдено единственное решение, или были рассмотрены все предложения.
\section{Виды и назначение переменных в Prolog. Примеры из задания. Почему использованы те или другие переменные (примеры из задания)?}
Ответ на первый вопрос см. пункт 1.4.